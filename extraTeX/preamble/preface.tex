\chapter*{Preface}

%\vspace{-50mm}

%{\Large Preface} \\

This book may be downloaded as a free PDF at \href{http://www.openintro.org}{\color{black}\textbf{openintro.org}}. %This download is free and need not be supplemented by the purchase of a paperback copy.
\vspace{3mm}

\noindent We hope readers will take away three ideas from this book in addition to forming a foundation of statistical thinking and methods.\vspace{-1mm}
\begin{enumerate}
\setlength{\itemsep}{0mm}
\item[(1)] Statistics is an applied field with a wide range of practical applications.
\item[(2)] You don't have to be a math guru to learn from real, interesting data.
\item[(3)] Data is messy, and statistical tools are imperfect. But, when you understand the strengths and weaknesses of these tools, you can use them to learn about the real~world.
\end{enumerate}
%This textbook is supplemented by many free, online resources to help students apply the methods they learn in this textbook and beyond.


\subsection*{Textbook overview}

The chapters of this book are as follows:
\begin{description}
\setlength{\itemsep}{0mm}
\item[1. Introduction to data.] Data structures, variables, summaries, graphics, and basic data collection techniques.
\item[2. Probability (special topic).] The basic principles of probability. An understanding of this chapter is not required for the main content in Chapters~\ref{modeling}-\ref{multipleAndLogisticRegression}.
\item[3. Distributions of random variables.] Introduction to the normal model and other key distributions.
\item[4. Foundations for inference.] General ideas for statistical inference in the context of estimating the population mean.
\item[5. Inference for numerical data.] Inference for one or two sample means using the normal model and $t$ distribution, and also comparisons of many means using ANOVA.
\item[6. Inference for categorical data.] Inference for proportions using the normal and chi-square distributions, as well as simulation and randomization techniques.
%\item[5. Large sample inference.] Inferential methods for one or two sample means and proportions using the normal model, and also contingency tables via chi-square.
%\item[6. Small sample inference.] Inference for means using the $t$ distribution, as well as simulation and randomization techniques for proportions.
\item[7. Introduction to linear regression.] An introduction to regression with two variables. Most of this chapter could be covered after Chapter~\ref{introductionToData}.
\item[8. Multiple and logistic regression.] An introduction to multiple regression and logistic regression for an accelerated course.
\end{description}

%This textbook was written to allow flexibility in choosing and ordering course topics. The material is divided into two pieces: main text and special topics. The goal of the main text is to allow people to move towards statistical inference and modeling sooner rather than later. Special topics, labeled in the table of contents and in section titles, may be added to a course as they arise naturally in the curriculum.

\emph{OpenIntro Statistics} was written to allow flexibility in choosing and ordering course topics. The material is divided into two pieces: main text and special topics. The main text has been structured to bring statistical inference and modeling closer to the front of a course. Special topics, labeled in the table of contents and in section titles, may be added to a course as they arise naturally in the curriculum.

%\emph{OpenIntro Statistics} would also serve as a helpful supplement in a course preparing students for the Advanced Placement Statistics exam, either though the textbook or use of the online resources outlined below. %We hope for this textbook to be deemed ``AP ready'' for future editions, which will require an application for approval from College Board. %The AP material would include the main and optional materials of Chapters~1-7 with the possible omission of Sections~1.8, 3.5, 4.7, 6.3-6.4, and~7.4.

\subsection*{Examples, exercises, and appendices}

Examples and within-chapter exercises throughout the textbook may be identified by their distinctive bullets:

\begin{example}{Large filled bullets signal the start of an example.}
Full solutions to examples are provided and often include an accompanying table or figure.
 \end{example}

\begin{exercise}
Large empty bullets signal to readers that an exercise has been inserted into the text for additional practice and guidance. Students may find it useful to fill in the bullet after understanding or successfully completing the exercise. Solutions are provided for all within-chapter exercises in footnotes.\footnote{Full solutions are located down here in the footnote!}
\end{exercise}

There are exercises at the end of each chapter that are useful for practice or homework assignments. Many of these questions have multiple parts, and odd-numbered questions include solutions in Appendix~\ref{eoceSolutions}. %These end-of-chapter exercises are also available online in a public question bank at \textbf{openintro.org}, and the available selection is constantly growing based on teacher contributions. Numbered citations in end-of-chapter exercises may be found in Appendix~B.

Probability tables for the normal, $t$, and chi-square distributions are in Appendix~\ref{distributionTables}, and PDF copies of these tables are also available from \href{http://www.openintro.org}{\color{black}\textbf{openintro.org}} for anyone to download, print, share, or modify.

\subsection*{OpenIntro, online resources, and getting involved}

OpenIntro is an organization focused on developing free and affordable education materials. \emph{OpenIntro Statistics}, our first project, is intended for introductory statistics courses at the high school through university levels.

%We encourage anyone learning or teaching statistics to visit \textbf{openintro.org} and get involved by using the many online resources, which are all free and rapidly expanding, or by creating new material. Students can test their knowledge with practice quizzes for each chapter, or try an application of concepts learned in each chapter using real data and the top-rated and free statistical software \emph{R}. Teachers can download the source for course materials, labs, slides, data sets, R figures, or create their own custom quizzes and problem sets for students to take on the website. Everyone is also welcome to download this textbook as a PDF or the book's source files to create a custom version of this textbook or to simply share a copy with a friend or on a website. All of these products are free, and we want to be clear that anyone is welcome to use these online tools and resources with or without this textbook as a companion.

We encourage anyone learning or teaching statistics to visit \href{http://www.openintro.org}{\color{black}\textbf{openintro.org}} and get involved. We also provide many free online resources, including free course software. 
%, or by creating new material. Students can test their knowledge with practice quizzes for each chapter, or try an application of concepts learned using real data. 
Data sets for this textbook are available on the website and through a companion R package.\footnote{Diez DM, Barr CD, \c{C}etinkaya-Rundel M. 2012. \texttt{openintro}: OpenIntro data sets and supplement functions. \urlwofont{http://cran.r-project.org/web/packages/openintro}.} All of these resources are free, and we want to be clear that anyone is welcome to use these online tools and resources with or without this textbook as a companion.

%Teachers can download the source files for this book, labs, data sets, or create their own custom quizzes and problem sets for students to take at \href{http://www.openintro.org}{\textbf{openintro.org}}. 
%Anyone can download a PDF or the source files of this textbook for modifying and sharing at \href{http://www.openintro.org}{\color{black}\textbf{openintro.org}}.

We value your feedback. If there is a particular component of the project you especially like or think needs improvement, we want to hear from you. You may find our contact information on the title page of this book or on the \href{http://www.openintro.org/about.php}{About} section of \href{http://www.openintro.org}{\color{black}\textbf{openintro.org}}.

\subsection*{Acknowledgements}

This project would not be possible without the dedication and volunteer hours of all those involved. No one has received any monetary compensation from this project, and we hope you will join us in extending a \emph{thank you} to all those volunteers below.

The authors would like to thank Andrew Bray, Meenal Patel, Yongtao Guan, Filipp Brunshteyn, Rob Gould, and Chris Pope for their involvement and contributions. %A special thank you to Andrew Bray, who has developed the R labs and diligently worked to translate end-of-chapter exercises to the project website. We also deeply appreciate the contribution of Meenal Patel, who has helped raise the professional profile of OpenIntro by designing a business system and website for the project. 
We are also very grateful to Dalene Stangl, Dave Harrington, Jan de Leeuw, Kevin Rader, and Philippe Rigollet for providing us with valuable feedback.

% Add note about exercise/example circles

% Add note about margin notes for notation


