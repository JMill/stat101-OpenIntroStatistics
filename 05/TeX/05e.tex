
\section{Exercises}

%__________________
\subsection{Paired data}

% 1

\eoce{\qt{Global warming, Part I} \label{warming} Is there strong evidence of global warming? Let's consider a small scale example, comparing how temperatures have changed in the US from 1968 to 2008. The daily high temperature reading on January 1 was collected in 1968 and 2008 for 51 randomly selected locations in the continental US. Then the difference between the two readings (temperature in 2008 - temperature in 1968) was calculated for each of the 51 different locations. The average of these 51 values was 1.1 degrees with a standard deviation of 4.9 degrees. We are interested in determining whether these data provide strong evidence of temperature warming in the continental US. 
\begin{parts}
\item Is there a relationship between the observations collected in 1968 and 2008? Or are the observations in the two groups independent? Explain.
\item Write hypotheses for this research in symbols and in words.
\item Check the conditions required to complete this test.
\item Calculate the test statistic and find the p-value.
\item What do you conclude? Interpret your conclusion in context.
\item What type of error might we have made? Explain in context what the error means.
\item Based on the results of this hypothesis test, would you expect a confidence interval for the average difference between the temperature measurements from 1968 and 2008 to include 0? Explain your reasoning.
\end{parts}
}{}

% 2

\eoce{\qt{High School and Beyond, Part I} The National Center of Education Statistics conducted a survey of high school seniors, collecting test data on reading, writing, and several other subjects. Here we examine a simple random sample of 200 students from this survey. Side-by-side box plots of reading and writing scores as well as a histogram of the differences in scores are shown below.
\begin{center}
\includegraphics[width=0.45\textwidth]{05/figures/eoce/hsb2/hsb2_read_write_box}
\includegraphics[width=0.51\textwidth]{05/figures/eoce/hsb2/hsb2_diff_hist}
\end{center}
\begin{parts}

\item Is there a clear difference in the average reading and writing scores?

\item Are the reading and writing scores of each student independent of each other?

\item Create hypotheses appropriate for the following research question: is there an evident difference in the average scores of students in the reading and writing exam?

\item Check the conditions required to complete this test.

\item The average observed difference in scores is $\bar{x}_{read-write} = -0.545$, and the standard deviation of the differences is 8.887 points. Do these data provide convincing evidence of a difference between the average scores on the two exams?
\item What type of error might we have made? Explain what the error means in the context of the application.
\item Based on the results of this hypothesis test, would you expect a confidence interval for the average difference between the reading and writing scores to include 0? Explain your reasoning.
\end{parts}
}{}

\textB{\pagebreak}

% 3

\eoce{\qt{Global warming, Part II} \label{hsb2} We considered the differences between the temperature readings in January 1 of 1968 and 2008 at 51 locations in the continental US in Exercise~\ref{warming}. The mean and standard deviation of the reported differences are 1.1 degrees and 4.9 degrees.
\begin{parts}
\item Calculate a 90\% confidence interval for the average difference between the temperature measurements between 1968 and 2008.
\item Interpret this interval in context.
\item Does the confidence interval provide convincing evidence that the temperature was higher in 2008 than in 1968 in the continental US? Explain.
\end{parts}
}{}

 % 4

\eoce{\qt{High school and beyond, Part II} We considered the differences between the reading and writing scores of a random sample of 200 students who took the High School and Beyond Survey in Exercise~\ref{hsb2}. The mean and standard deviation of the differences are $\bar{x}_{read-write} = -0.545$ and 8.887 points.
\begin{parts}
\item Calculate a 95\% confidence interval for the average difference between the reading and writing scores of all students.
\item Interpret this interval in context.
\item Does the confidence interval provide convincing evidence that there is a real difference in the average scores? Explain.
\end{parts}
}{}

% 5

\eoce{\qt{Gifted children} Researchers collected a simple random sample of 36 children who had been identified as gifted in a large city. The following histograms show the distributions of the IQ scores of mothers and fathers of these children. Also provided are some sample statistics.\footfullcite{Graybill:1994}

\begin{center}
\includegraphics[width=0.9\textwidth]{05/figures/eoce/gifted/gifted_IQ_hist} \\[2mm]
{\small
\begin{tabular}{r | c c c}
		& Mother	& Father	& Diff. \\
\hline
Mean	& 118.2 	& 114.8	& 3.4 \\
SD		& 6.5		& 3.5		& 7.5 \\
n		& 36		& 36		& 36
\end{tabular}
}
\end{center}

\begin{parts}
\item Are the IQs of mothers and the IQs of fathers in this data set related? Explain.
\item Conduct a hypothesis test to evaluate if the scores are equal on average. Make sure to clearly state your hypotheses, check the relevant conditions, and state your conclusion in the context of the data.
\end{parts}
}{}

% 6

\eoce{\qtq{Paired or not} In each of the following scenarios, determine if the data are paired.
\begin{parts}
\item We would like to know if Intel's stock and Southwest Airlines' stock have similar rates of return. To find out, we take a random sample of 50 days for Intel's stock and another random sample of 50 days for Southwest's stock.
\item We randomly sample 50 items from Target stores and note the price for each. Then we visit Walmart and collect the price for each of those same 50 items.
\item A school board would like to determine whether there is a difference in average SAT scores for students at one high school versus another high school in the district. To check, they take a simple random sample of 100 students from each high school.
\end{parts}
}{}

\textB{\pagebreak}


%__________________
\subsection{Difference of two means}

% 7

\eoce{\qt{Math scores of 13 year olds, Part I} \label{mathScore} The National Assessment of Educational Progress tested a simple random sample of 1,000 thirteen year old students in both 2004 and 2008 (two separate simple random samples). The average and standard deviation in 2004 were 257 and 39, respectively. In 2008, the average and standard deviation were 260 and 38, respectively. Calculate a 90\% confidence interval for the change in average scores from 2004 to 2008, and interpret this interval in the context of the application. (Reminder: check conditions.)  \footfullcite{web:naep}
}{}

% 8

\eoce{\qt{Work hours and education, Part I} \label{workDegCI2Samp} The General Social Survey collects data on demographics, education, and work, among many other characteristics of US residents. The histograms below display the distributions of hours worked per week for two education groups: those with and without a college degree.\footfullcite{data:gss:2010}
Suppose we want to estimate the average difference between the number of hours worked per week by all Americans with a college degree and those without a college degree. Summary information for each group is shown in the tables.

\noindent\begin{minipage}[c]{0.65\textwidth}
\begin{center}
\includegraphics[width=\textwidth]{05/figures/eoce/workDeg/workDeg_work_edu_hist}
\end{center}
\end{minipage}
\begin{minipage}[c]{0.35\textwidth}
\begin{tabular}{l c}
\hline
		& College degree	\\
\hline
Mean 	& 41.8 hrs		\\
SD 		& 15.1 hrs		\\
n 		& 505		\\
		& \\
\hline
		& No college degree \\
\hline
Mean	& 39.4 hrs \\
SD		& 15.1 hrs \\
n		& 667
\end{tabular}
\end{minipage}

\begin{parts}
\item What is the parameter of interest, and what is the point estimate?
\item Are conditions satisfied for estimating this difference using a confidence interval?
\item Create a 95\% confidence interval for the difference in number of hours worked between the two groups, and interpret the interval in context.
\item Can you think of any real world justification for your results? (\textit{Note:} There isn't a single correct answer to this question.)
\end{parts}
}{} 

% 9

\eoce{\qt{Math scores of 13 year olds, Part II} Exercise~\ref{mathScore} provides data on the average math scores from tests conducted by the National Assessment of Educational Progress in 2004 and 2008. Two separate simple random samples were taken in each of these years. The average and standard deviation in 2004 were 257 and 39, respectively. In 2008, the average and standard deviation were 260 and 38, respectively.
\begin{parts}
\item Do these data provide strong evidence that the average math score for 13 year old students has changed from 2004 to 2008? Use a 10\% significance level.
\item It is possible that your conclusion in part (a) is incorrect. What type of error is possible for this conclusion? Explain.
\item Based on your hypothesis test, would you expect a 90\% confidence interval to contain the null value? Explain.\end{parts}
}{}

\textB{\newpage}

% 10

\eoce{\qt{Work hours and education, Part II} \label{workDegHT2Samp} The General Social Survey described in Exercise~\ref{workDegCI2Samp} included random samples from two groups: US residents with a college degree and US residents without a college degree. For the 505 sampled US residents with a college degree, the average number of hours worked each week was 41.8 hours with a standard deviation of 15.1 hours. For those 667 without a degree, the mean was 39.4 hours with a standard deviation of 15.1 hours. Conduct a hypothesis test to check for a difference in the average number of hours worked for the two groups.
}{}

% 11

\eoce{\qtq{Does the Paleo diet work} The Paleo diet allows only for foods that humans typically consumed over the last 2.5 million years, excluding those agriculture-type foods that arose during the last 10,000 years or so. Researchers randomly divided 500 volunteers into two equal-sized groups. One group spent 6 months on the Paleo diet. The other group received a pamphlet about controlling portion sizes. Randomized treatment assignment was performed, and at the beginning of the study, the average difference in weights between the two groups was about 0. After the study, the Paleo group had lost on average 7 pounds with a standard deviation of 20 pounds while the control group had lost on average 5 pounds with a standard deviation of 12 pounds.
\begin{parts}
\item The 95\% confidence interval for the difference between the two population parameters (Paleo - control) is given as (-0.891, 4.891). Interpret this interval in the context of the data.
\item Based on this confidence interval, do the data provide convincing evidence that the Paleo diet is more effective for weight loss than the pamphlet (control)? Explain your reasoning.
\item Without explicitly performing the hypothesis test, do you think that if the Paleo group had lost 8 instead of 7 pounds on average, and everything else was the same, the results would then indicate a significant difference between the treatment and control groups? Explain your reasoning.
\end{parts}
}{}

% 12

\eoce{\qt{Weight gain during pregnancy} In 2004, the state of North Carolina released to the public a large data set containing information on births recorded in this state. This data set has been of interest to medical researchers who are studying the relationship between habits and practices of expectant mothers and the birth of their children. The following histograms show the distributions of weight gain during pregnancy by 867 younger moms (less than 35 years old) and 133 mature moms (35 years old and over) who have been randomly sampled from this large data set. The average weight gain of younger moms is 30.56 pounds, with a standard deviation of 14.35 pounds, and the average weight gain of mature moms is 28.79 pounds, with a standard deviation of 13.48 pounds. Calculate a 95\% confidence interval for the difference between the average weight gain of younger and mature moms. Also comment on whether or not this interval provides strong evidence that there is a significant difference between the two population means.
\begin{center}
\includegraphics[width=0.9\textwidth]{05/figures/eoce/ncbirths/ncbirths_hist}
\end{center}
}{}

% 13

\eoce{\qt{Body fat in women and men} The third National Health and Nutrition Examination Survey collected body fat percentage (BF) data from 13,601 subjects whose ages are 20 to 80. A summary table for these data is given below. Note that BF is given as \textit{mean $\pm$ standard error}. Construct a 95\% confidence interval for the difference in average body fat percentages between men and women, and explain the meaning of this interval.\footfullcite{Romero:2008} 
\begin{center}
\begin{tabular}{l c c }
\hline
Gender	& n		& BF (\%)				\\
\hline
Men		& 6,580	& 23.9 $\pm$ 0.07	 \\
Women	& 7,021	& 35.0 $\pm$ 0.09	 \\
\hline
\end{tabular}
\end{center}
}{}

% 14

\eoce{\qt{Child care hours, Part I} \label{china} The China Health and Nutrition Survey aims to examine the effects of the health, nutrition, and family planning policies and programs implemented by national and local governments. One of the variables collected on the survey is the number of hours parents spend taking care of children in their household under age 6 (feeding, bathing, dressing, holding, or watching them). In 2006, 487 females and 312 males were surveyed for this question. On average, females reported spending 31 hours with a standard deviation of 31 hours, and males reported spending 16 hours with a standard deviation of 21 hours. Calculate a 95\% confidence interval for the difference between the average number of hours Chinese males and females spend taking care of their children under age 6. Also comment on whether this interval suggests a significant difference between the two population parameters. You may assume that conditions for inference are satisfied.\footfullcite{data:china}
}{}


%__________________
\subsection{One-sample means with the $t$ distribution}

% 15

\eoce{\qt{Identify the critical $t$} An independent random sample is selected from an approximately normal population with unknown standard deviation. Find the degrees of freedom and the critical $t$ value (t$^\star$) for the given sample size and confidence level.\vspace{-3mm}
\begin{multicols}{2}
\begin{parts}
\item $n = 6$, CL = 90\%
\item $n = 21$, CL = 98\%
\item $n = 29$, CL = 95\%
\item $n = 12$, CL = 99\%
\end{parts}
\end{multicols}
}{}

% 16

\eoce{\qt{Working backwards, Part I} A 90\% confidence interval for a population mean is (65,77). The population distribution is approximately normal and the population standard deviation is unknown. This confidence interval is based on a simple random sample of 25 observations. Calculate the sample mean, the margin of error, and the sample standard deviation.
}{}

% 17

\eoce{\qt{Working backwards, Part II} A 95\% confidence interval for a population mean, $\mu$, is given as (18.985, 21.015). This confidence interval is based on a simple random sample of 36 observations. Calculate the sample mean and standard deviation. Assume that all conditions necessary for inference are satisfied. Use the $t$ distribution in any calculations.
}{}

% 18

\eoce{\qt{Find the p-value} An independent random sample is selected from an approximately normal population with an unknown standard deviation. Find the p-value for the given set of hypotheses and $T$ test statistic. Also determine if the null hypothesis would be rejected at $\alpha = 0.05$.\vspace{-3mm}
\begin{multicols}{2}
\begin{parts}
\item $H_A: \mu > \mu_0 $, $n = 11$, $T = 1.91$
\item $H_A: \mu < \mu_0 $, $n = 17$, $T = -3.45$
\item $H_A: \mu \ne \mu_0 $, $n = 7$, $T = 0.83$
\item $H_A: \mu > \mu_0 $, $n = 28$, $T = 2.13$
\end{parts}
\end{multicols}
}{}

\textB{\pagebreak}

% 19

\eoce{\qt{Sleep habits of New Yorkers} \label{NewYorkSleep} New York is known as ``the city that never sleeps". A random sample of 25 New Yorkers were asked how much sleep they get per night. Statistical summaries of these data are shown below. Do these data provide strong evidence that New Yorkers sleep less than 8 hours a night on average?
\begin{center}
\begin{tabular}{rrrrrr}
 \hline
n 	& $\bar{x}$	& s		& min 	& max \\ 
 \hline
25 	& 7.73 		& 0.77 	& 6.17 	& 9.78 \\ 
  \hline
\end{tabular}
\end{center}

\begin{parts}
\item Write the hypotheses in symbols and in words.
\item Check conditions, then calculate the test statistic, $T$, and the associated degrees of freedom.
\item Find and interpret the p-value in this context. Drawing a picture may be helpful.
\item What is the conclusion of the hypothesis test?
\item If you were to construct a 90\% confidence interval that corresponded to this hypothesis test, would you expect 8 hours to be in the interval?
\end{parts}
}{}

% 20

\eoce{\qt{Fuel efficiency of Prius} Fueleconomy.gov, the official US government source for fuel economy information, allows users to share gas mileage information on their vehicles. The histogram below shows the distribution of gas mileage in miles per gallon (MPG) from 14 users who drive a 2012 Toyota Prius. The sample mean is 53.3 MPG and the standard deviation is 5.2 MPG. Note that these data are user estimates and since the source data cannot be verified, the accuracy of these estimates are not guaranteed.\footfullcite{data:prius}
\begin{center}
\includegraphics[width=0.6\textwidth]{05/figures/eoce/prius/prius_hist}
\end{center}
\begin{parts}
\item We would like to use these data to evaluate the average gas mileage of all 2012 Prius drivers. Do you think this is reasonable? Why or why not?
\item The EPA claims that a 2012 Prius gets 50 MPG (city and highway mileage combined). Do these data provide strong evidence against this estimate for drivers who participate on fueleconomy.gov? Note any assumptions you must make as you proceed with the test.
\item Calculate a 95\% confidence interval for the average gas mileage of a 2012 Prius by drivers who participate on fueleconomy.gov.
\end{parts}
}{}

% 21

\eoce{\qt{Find the mean} You are given the following hypotheses:
\begin{align*}
H_0&: \mu = 60 \\
H_A&: \mu < 60
\end{align*}
We know that the sample standard deviation is 8 and the sample size is 20. For what sample mean would the p-value be equal to 0.05? Assume that all conditions necessary for inference are satisfied.
}{}

% 22

\eoce{\qt{$t^\star$ vs. $z^\star$} For a given confidence level, $t^{\star}_{df}$ is larger than $z^{\star}$. Explain how $t^{*}_{df}$ being slightly larger than $z^{*}$ affects the width of the confidence interval.
}{}



%__________________
\subsection{Comparing many means with ANOVA}

% 37

\eoce{\qt{Chicken diet and weight} If we're raising chicks, we could compare the effects of two types of feed at a time. A better analysis would first consider all feed types at once: casein, horsebean, linseed, meat meal, soybean, and sunflower. The ANOVA output below can be used to test for differences between the average weights of chicks on different diets.
\begin{center}
\begin{tabular}{lrrrrr}
\hline
 		& Df 	& Sum Sq		& Mean Sq 	& F value 	& Pr($>$F) \\ 
\hline
feed 		& 5 	& 231,129.16 	& 46,225.83 	& 15.36 	& 0.0000 \\ 
Residuals	& 65 & 195,556.02 	& 3,008.55	&  		&  \\ 
\hline
%\multicolumn{6}{r}{$s_{pooled} = 55.85$ on $df=65$}
\end{tabular}
\end{center}
Conduct a hypothesis test to determine if these data provide convincing evidence that the average weight of chicks varies across some (or all) groups. Make sure to check relevant conditions. Figures and summary statistics are shown below.

\begin{minipage}[c]{0.65\textwidth}
\begin{center}
\includegraphics[width= \textwidth]{05/figures/eoce/chicks/chicks_box}
\end{center}
\end{minipage}
\begin{minipage}[c]{0.35\textwidth}
{\footnotesize\begin{tabular}{l c c c}
\hline
       		& Mean		& SD		& n \\
\hline
casein  		& 323.58 		& 64.43	& 12 \\
horsebean 	& 160.20 		& 38.63	& 10 \\
linseed  		& 218.75 		& 52.24	& 12 \\
meatmeal 	& 276.91 		& 64.90	& 11 \\
soybean  		& 246.43 		& 54.13	& 14 \\
sunflower 		& 328.92 		& 48.84	& 12 \\
\hline
\end{tabular}}
\end{minipage} 
}{}

\textB{\pagebreak}

% 38

\eoce{\qt{Student performance across discussion sections} A professor who teaches a large introductory statistics class (197 students) with eight discussion sections would like to test if student performance differs by discussion section, where each discussion section has a different teaching assistant. The summary table below shows the average final exam score for each discussion section as well as the standard deviation of scores and the number of students in each section.
\begin{center}
\begin{tabular}{rrrrrrrrr}
  \hline
 			& Sec 1 & Sec 2 & Sec 3 & Sec 4 & Sec 5 & Sec 6 & Sec 7 & Sec 8 \\ 
  \hline
$n_i$		& 33 & 19 & 10 & 29 & 33 & 10 & 32 & 31 \\ 
$\bar{x}_i$	& 92.94 & 91.11 & 91.80 & 92.45 & 89.30 & 88.30 & 90.12 & 93.35 \\ 
$s_i$ 		& 4.21 & 5.58 & 3.43 & 5.92 & 9.32 & 7.27 & 6.93 & 4.57 \\ 
   \hline
\end{tabular}
\end{center}
The ANOVA output below can be used to test for differences between the average scores from the different discussion sections.
\begin{center}
\begin{tabular}{lrrrrr}
\hline
 			& Df 		& Sum Sq & Mean Sq 	& F value & Pr($>$F) \\ 
\hline
section 		& 7 		& 525.01 	& 75.00 		& 1.87 	& 0.0767 \\ 
Residuals 	& 189	& 7584.11	& 40.13 		&  		&  \\ 
\hline
\end{tabular}
\end{center}
Conduct a hypothesis test to determine if these data provide convincing evidence that the average score varies across some (or all) groups. Check conditions and describe any assumptions you must make to proceed with the test.
}{}

% 39

\eoce{\qt{Coffee, depression, and physical activity} \label{coffeeDepression} Caffeine is the world's most widely used stimulant, with approximately 80\% consumed in the form of coffee. Participants in a study investigating the relationship between coffee consumption and exercise were asked to report the number of hours they spent per week on moderate (e.g., brisk walking) and vigorous (e.g., strenuous sports and jogging) exercise. Based on these data the researchers estimated the total hours of metabolic equivalent tasks (MET) per week, a value always greater than 0. The table below gives summary statistics of MET for women in this study based on the amount of coffee consumed.\footfullcite{Lucas:2011}
 
\begin{adjustwidth}{-4em}{-4em}

\begin{center}
\begin{tabular}{l  r  r  r  r  r  r}
\multicolumn{1}{c}{}	& \multicolumn{5}{c}{\textit{Caffeinated coffee consumption}} \\
\cline{2-6}
				& $\le$ 1 cup/week	& 2-6 cups/week	& 1 cup/day	& 2-3 cups/day & $\ge$ 4 cups/day & Total	\\
\hline
Mean			& 18.7	& 19.6	& 19.3	& 18.9	& 17.5 			  \\
SD				& 21.1	& 25.5	& 22.5	& 22.0	& 22.0 \\
n				& 12,215	& 6,617 		& 17,234	& 12,290	& 2,383 	& 50,739 \\
\hline
\end{tabular}
\end{center}
\end{adjustwidth}

\begin{parts}

\item Write the hypotheses for evaluating if the average physical activity level varies among the different levels of coffee consumption.

\item Check conditions and describe any assumptions you must make to proceed with the test.

\item Below is part of the output associated with this test. Fill in the empty cells.

\begin{center}
\renewcommand{\arraystretch}{1.25}
\begin{tabular}{lrrrrr}
  \hline
 			& Df 	& Sum Sq		& Mean Sq	& F value	& Pr($>$F) \\ 
  \hline
coffee	 	& \fbox{\textcolor{white}{{\footnotesize XXXXX}}}	 & \fbox{\textcolor{white}{{\footnotesize XXXXX}}} 		& \fbox{\textcolor{white}{{\footnotesize XXXXX}}} 			& \fbox{\textcolor{white}{{\footnotesize XXXXX}}} 	& 0.0003 \\ 
Residuals		& \fbox{\textcolor{white}{{\footnotesize XXXXX}}} & 25,564,819 	& \fbox{\textcolor{white}{{\footnotesize  XXXXX}}} 			&  		&  \\ 
   \hline
Total			& \fbox{\textcolor{white}{{\footnotesize XXXXX}}} &25,575,327
\end{tabular}
\end{center}

\item What is the conclusion of the test?

\end{parts}
}{}

\textB{\pagebreak}

% 40

\eoce{\qt{Work hours and education, Part III} In Exercises~\ref{workDegCI2Samp} and \ref{workDegHT2Samp} you worked with data from the General Social Survey in order to compare the average number of hours worked per week by US residents with and without a college degree. However, this analysis didn't take advantage of the original data which contained more accurate information on educational attainment (less than high school, high school, junior college, Bachelor's, and graduate school). Using ANOVA, we can consider educational attainment levels for all 1,172 respondents at once instead of re-categorizing them into two groups. Below are the distributions of hours worked by educational attainment and relevant summary statistics that will be helpful in carrying out this analysis.
\begin{center}

\begin{tabular}{l  r  r  r  r  r  r}
\multicolumn{1}{c}{}	& \multicolumn{5}{c}{\textit{Educational attainment}} \\
\cline{2-6}
				& Less than HS 	& HS		& Jr Coll	& Bachelor's & Graduate & Total	\\
\hline
Mean			& 38.67			& 39.6	& 41.39	& 42.55	& 40.85 	& 40.45			  \\
SD				& 15.81			& 14.97	& 18.1	& 13.62	& 15.51	& 15.17		 \\
n				& 121			& 546 	& 97		& 253	& 155 	& 1,172 \\
\hline
\end{tabular}

\includegraphics[width=\textwidth]{05/figures/eoce/workDeg/workDeg_work_edu_raw_box}
\end{center}
\begin{parts}
\item Write hypotheses for evaluating whether the average number of hours worked varies across the five groups.
\item Check conditions and describe any assumptions you must make to proceed with the test.
\item Below is part of the output associated with this test. Fill in the empty cells.

\begin{center}
\renewcommand{\arraystretch}{1.25}
\begin{tabular}{lrrrrr}
  \hline
 			& Df 	& Sum Sq									& Mean Sq										& F value										& Pr($>$F) \\ 
  \hline
degree	 	& \fbox{\textcolor{white}{{\footnotesize XXXXX}}}	 	& \fbox{\textcolor{white}{{\footnotesize XXXXX}}} 		& 501.54	& \fbox{\textcolor{white}{{\footnotesize XXXXX}}} 	& 0.0682 \\ 
Residuals		& \fbox{\textcolor{white}{{\footnotesize XXXXX}}} & 267,382 	& \fbox{\textcolor{white}{{\footnotesize  XXXXX}}} 			&  		&  \\ 
   \hline
Total			& \fbox{\textcolor{white}{{\footnotesize XXXXX}}} &\fbox{\textcolor{white}{{\footnotesize XXXXX}}}
\end{tabular}
\end{center}

\item What is the conclusion of the test?

\end{parts}
}{}

\textB{\newpage}

% 41

\eoce{\qt{GPA and major} Undergraduate students taking an introductory statistics course at Duke University conducted a survey about GPA and major. The side-by-side box plots show the distribution of GPA among three groups of majors. Also provided is the ANOVA output.

\begin{center}
\includegraphics[width=0.8\textwidth]{05/figures/eoce/gpaMajor/gpaMajor_box}
\end{center}
\begin{center}
\begin{tabular}{lrrrrr}
  \hline
 & Df & Sum Sq & Mean Sq & F value & Pr($>$F) \\ 
  \hline
major & 2 & 0.03 & 0.02 & 0.21 & 0.8068 \\ 
  Residuals & 195 & 15.77 & 0.08 &  &  \\ 
   \hline
\end{tabular}
\end{center}
\begin{parts}
\item Write the hypotheses for testing for a difference between average GPA across majors.
\item What is the conclusion of the hypothesis test?
\item How many students answered these questions on the survey, i.e. what is the sample size?
\end{parts}
}{}

% 42

\eoce{\qt{Child care hours, Part II} Exercise~\ref{china} introduces the  China Health and Nutrition Survey which, among other things, collects information on number of hours Chinese parents spend taking care of their children under age 6. The side by side box plots below show the distribution of this variable by educational attainment of the parent. Also provided below is the ANOVA output for comparing average hours across educational attainment categories.
\begin{center}
\includegraphics[width=\textwidth]{05/figures/eoce/china/china_edu_box}
\end{center}
\begin{center}
\begin{tabular}{lrrrrr}
  \hline
 & Df & Sum Sq & Mean Sq & F value & Pr($>$F) \\ 
  \hline
education & 4 & 4142.09 & 1035.52 & 1.26 & 0.2846 \\ 
  Residuals & 794 & 653047.83 & 822.48 &  &  \\ 
   \hline
\end{tabular}
\end{center}
\begin{parts}
\item Write the hypotheses for testing for a difference between the average number of hours spent on child care across educational attainment levels.
\item What is the conclusion of the hypothesis test?
\end{parts}
}{}

\textB{\pagebreak}

% 43

\eoce{\qt{True or false, Part II} Determine if the following statements are true or false in ANOVA, and explain your reasoning for statements you identify as false.
\begin{parts}
\item As the number of groups increases, the modified significance level for pairwise tests increases as well.
\item As the total sample size increases, the degrees of freedom for the residuals increases as well.
\item The constant variance condition can be somewhat relaxed when the sample sizes are relatively consistent across groups.
\item The independence assumption can be relaxed when the total sample size is large.
\end{parts}
}{}

% 44

\eoce{\qt{True or false, Part III} Determine if the following statements are true or false, and explain your reasoning for statements you identify as false.

If the null hypothesis that the means of four groups are all the same is rejected using ANOVA at a 5\% significance level, then ...
\begin{parts}
\item we can then conclude that all the means are different from one another.
\item the standardized variability between groups is higher than the standardized variability within groups.
\item the pairwise analysis will identify at least one pair of means that are significantly different.
\item the appropriate $\alpha$ to be used in pairwise comparisons is 0.05 / 4 = 0.0125 since there are four groups.
\end{parts}
}{}
